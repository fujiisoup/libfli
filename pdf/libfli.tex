\documentclass{article}
\usepackage{docxx}
\begin{document}
\pagebreak
\sloppy
\cxxAuthor{
Finger Lakes Instrumentation \\
Copyright (c) 2000-2002 Finger Lakes Instrumentation (FLI), LLC. \\
All rights reserved.
\strut}
\cxxVersion{
1.40
\strut}
\cxxTitle{}
        {FLI Software Development Library}
        {}
        {Windows and Linux support for FLI CCD cameras, filter wheels, and focusers. }
        {}
\begin{cxxContents}
\cxxContentsEntry{1}{Introduction}{}
\cxxContentsEntry{2}{Library Defined Types }{}
\cxxContentsEntry{3}{Library Functions }{}
\end{cxxContents}
\clearpage\pagebreak
\begin{cxxdoc}


\end{cxxdoc}
\begin{cxxentry}
{}
        {Introduction}
        {}
        {}
        {1}
\begin{cxxdoc}

This library provides a core set of functions for programming
FLI CCD cameras, filter wheels, and focusers under Windows and
Linux.  The type definitions, function prototypes, and
definitions/enumerations of constant values used by library
functions are spcified in \texttt{libfli.h}.  All library
functions return zero on sucessful completion, and non-zero if
an error occurred.  The exact nature of an error can be found by
treating the negative of a function's return value as a system
error code, for example:

\begin{verbatim}
      if ((err = FLIOpen(&dev, name, domain)))
      {
        fprintf(stderr, "Error FLIOpen: %s\n", strerror((int)-err));
        exit(1);
      }
      \end{verbatim}

\end{cxxdoc}
\end{cxxentry}
\begin{cxxentry}
{}
        {Library Defined Types }
        {}
        {}
        {2}
\begin{cxxnames}
\cxxitem{\#define}
        {FLI\_INVALID\_DEVICE}
        {(-1)}
        { An opaque handle used by library functions to refer to FLI hardware}
        {2.1}
\cxxitem{typedef\ \ long}
        {flidomain\_t}
        {}
        { The domain of an FLI device.}
        {2.2}
\cxxitem{typedef\ \ long}
        {fliframe\_t}
        {}
        { The frame type for an FLI CCD camera device.}
        {2.3}
\cxxitem{typedef\ \ long}
        {flibitdepth\_t}
        {}
        { The gray-scale bit depth for an FLI camera device.}
        {2.4}
\cxxitem{typedef\ \ long}
        {flishutter\_t}
        {}
        { Type used for shutter operations for an FLI camera device.}
        {2.5}
\cxxitem{typedef\ \ long}
        {flibgflush\_t}
        {}
        { Type used for background flush operations for an FLI camera device.}
        {2.6}
\cxxitem{typedef\ \ long}
        {flichannel\_t}
        {}
        { Type used to determine which temperature channel to read.}
        {2.7}
\cxxitem{typedef\ \ long}
        {flidebug\_t}
        {}
        { Type specifying library debug levels.}
        {2.8}
\end{cxxnames}
\begin{cxxmacro}
{\#define}
        {FLI\_INVALID\_DEVICE}
        {(-1)}
        { An opaque handle used by library functions to refer to FLI hardware}
        {2.1}
\begin{cxxdoc}

An opaque handle used by library functions to refer to FLI
hardware
\end{cxxdoc}
\end{cxxmacro}
\begin{cxxentry}
{typedef\ \ long}
        {flidomain\_t}
        {}
        { The domain of an FLI device.}
        {2.2}
\cxxSee{FLIOpen
\\
FLIList\strut}
\begin{cxxdoc}

The domain of an FLI device.  This consists of a bitwise ORed
combination of interface method and device type.  Valid interfaces
are \texttt{FLIDOMAIN\_PARALLEL\_PORT}, \texttt{FLIDOMAIN\_USB},
\texttt{FLIDOMAIN\_SERIAL}, and \texttt{FLIDOMAIN\_INET}.  Valid
device types are \texttt{FLIDEVICE\_CAMERA},
\texttt{FLIDOMAIN\_FILTERWHEEL}, and \texttt{FLIDOMAIN\_FOCUSER}.


\end{cxxdoc}
\end{cxxentry}
\begin{cxxentry}
{typedef\ \ long}
        {fliframe\_t}
        {}
        { The frame type for an FLI CCD camera device.}
        {2.3}
\cxxSee{FLISetFrameType\strut}
\begin{cxxdoc}

The frame type for an FLI CCD camera device.  Valid frame types are
\texttt{FLI\_FRAME\_TYPE\_NORMAL} and \texttt{FLI\_FRAME\_TYPE\_DARK}.


\end{cxxdoc}
\end{cxxentry}
\begin{cxxentry}
{typedef\ \ long}
        {flibitdepth\_t}
        {}
        { The gray-scale bit depth for an FLI camera device.}
        {2.4}
\cxxSee{FLISetBitDepth\strut}
\begin{cxxdoc}

The gray-scale bit depth for an FLI camera device.  Valid bit
depths are \texttt{FLI\_MODE\_8BIT} and \texttt{FLI\_MODE\_16BIT}.


\end{cxxdoc}
\end{cxxentry}
\begin{cxxentry}
{typedef\ \ long}
        {flishutter\_t}
        {}
        { Type used for shutter operations for an FLI camera device.}
        {2.5}
\cxxSee{FLIControlShutter\strut}
\begin{cxxdoc}

Type used for shutter operations for an FLI camera device.  Valid
shutter types are \texttt{FLI\_SHUTTER\_CLOSE},
\texttt{FLI\_SHUTTER\_OPEN},
\texttt{FLI\_SHUTTER\_EXTERNAL\_TRIGGER},
\texttt{FLI\_SHUTTER\_EXTERNAL\_TRIGGER\_LOW}, and
\texttt{FLI\_SHUTTER\_EXTERNAL\_TRIGGER\_HIGH}.


\end{cxxdoc}
\end{cxxentry}
\begin{cxxentry}
{typedef\ \ long}
        {flibgflush\_t}
        {}
        { Type used for background flush operations for an FLI camera device.}
        {2.6}
\cxxSee{FLIControlBackgroundFlush\strut}
\begin{cxxdoc}

Type used for background flush operations for an FLI camera device.  Valid
bgflush types are \texttt{FLI\_BGFLUSH\_STOP} and
\texttt{FLI\_BGFLUSH\_START}.


\end{cxxdoc}
\end{cxxentry}
\begin{cxxentry}
{typedef\ \ long}
        {flichannel\_t}
        {}
        { Type used to determine which temperature channel to read.}
        {2.7}
\cxxSee{FLIReadTemperature\strut}
\begin{cxxdoc}

Type used to determine which temperature channel to read.  Valid
channel types are \texttt{FLI\_TEMPERATURE\_INTERNAL} and
\texttt{FLI\_TEMPERATURE\_EXTERNAL}.


\end{cxxdoc}
\end{cxxentry}
\begin{cxxentry}
{typedef\ \ long}
        {flidebug\_t}
        {}
        { Type specifying library debug levels.}
        {2.8}
\cxxSee{FLISetDebugLevel\strut}
\begin{cxxdoc}

Type specifying library debug levels.  Valid debug levels are
\texttt{FLIDEBUG\_NONE}, \texttt{FLIDEBUG\_INFO},
\texttt{FLIDEBUG\_WARN}, and \texttt{FLIDEBUG\_FAIL}.


\end{cxxdoc}
\end{cxxentry}
\end{cxxentry}
\begin{cxxentry}
{}
        {Library Functions }
        {}
        {}
        {3}
\begin{cxxnames}
\cxxitem{LIBFLIAPI}
        {FLICancelExposure}
        {(flidev\_t\ dev)}
        { Cancel an exposure for a given camera.}
        {3.1}
\cxxitem{LIBFLIAPI}
        {FLIEndExposure}
        {(flidev\_t\ dev)}
        { End an exposure for a given camera.}
        {3.2}
\cxxitem{LIBFLIAPI}
        {FLITriggerExposure}
        {(flidev\_t\ dev)}
        { Trigger an exposure that is awaiting an external trigger.}
        {3.3}
\cxxitem{LIBFLIAPI}
        {FLIClose}
        {(flidev\_t\ dev)}
        { Close a handle to a FLI device  }
        {3.4}
\cxxitem{LIBFLIAPI}
        {FLIGetArrayArea}
        {(flidev\_t\ dev,\ long*\ ul\_x,\ long*\ ul\_y,\ long*\ lr\_x,\ long*\ lr\_y)}
        { Get the array area of the given camera.}
        {3.5}
\cxxitem{LIBFLIAPI}
        {FLIFlushRow}
        {(flidev\_t\ dev,\ long\ rows,\ long\ repeat)}
        { Flush rows of a given camera.}
        {3.6}
\cxxitem{LIBFLIAPI}
        {FLIGetFWRevision}
        {(flidev\_t\ dev,\ long*\ fwrev)}
        { Get firmware revision of a given device  }
        {3.7}
\cxxitem{LIBFLIAPI}
        {FLIGetHWRevision}
        {(flidev\_t\ dev,\ long*\ hwrev)}
        { Get the hardware revision of a given device  }
        {3.8}
\cxxitem{LIBFLIAPI}
        {FLIGetLibVersion}
        {(char*\ ver,\ size\_t\ len)}
        { Get the current library version.}
        {3.9}
\cxxitem{LIBFLIAPI}
        {FLIGetSerialString}
        {(flidev\_t\ dev,\ char*\ serial,\ size\_t\ len)}
        { Get the serial string of a given device.}
        {3.10}
\cxxitem{LIBFLIAPI}
        {FLIGetModel}
        {(flidev\_t\ dev,\ char*\ model,\ size\_t\ len)}
        { Get the model of a given device.}
        {3.11}
\cxxitem{LIBFLIAPI}
        {FLIGetPixelSize}
        {(flidev\_t\ dev,\ double*\ pixel\_x,\ double*\ pixel\_y)}
        { Find the dimensions of a pixel in the array of the given device  }
        {3.12}
\cxxitem{LIBFLIAPI}
        {FLIGetVisibleArea}
        {(flidev\_t\ dev,\ long*\ ul\_x,\ long*\ ul\_y,\ long*\ lr\_x,\ long*\ lr\_y)}
        { Get the visible area of the given camera.}
        {3.13}
\cxxitem{LIBFLIAPI}
        {FLIOpen}
        {(flidev\_t*\ dev,\ char*\ name,\ flidomain\_t\ domain)}
        { Get a handle to an FLI device.}
        {3.14}
\cxxitem{LIBFLIAPI}
        {FLISetDebugLevel}
        {(char*\ host,\ flidebug\_t\ level)}
        { Enable debugging of API operations and communications.}
        {3.15}
\cxxitem{LIBFLIAPI}
        {FLISetExposureTime}
        {(flidev\_t\ dev,\ long\ exptime)}
        { Set the exposure time for a camera.}
        {3.16}
\cxxitem{LIBFLIAPI}
        {FLISetHBin}
        {(flidev\_t\ dev,\ long\ hbin)}
        { Set the horizontal bin factor for a given camera.}
        {3.17}
\cxxitem{LIBFLIAPI}
        {FLISetFrameType}
        {(flidev\_t\ dev,\ fliframe\_t\ frametype)}
        { Set the frame type for a given camera.}
        {3.18}
\cxxitem{LIBFLIAPI}
        {FLIGetCoolerPower}
        {(flidev\_t\ dev,\ double*\ power)}
        { Get the cooler power level.}
        {3.19}
\cxxitem{LIBFLIAPI}
        {FLISetImageArea}
        {(flidev\_t\ dev,\ long\ ul\_x,\ long\ ul\_y,\ long\ lr\_x,\ long\ lr\_y)}
        { Set the image area for a given camera.}
        {3.20}
\cxxitem{LIBFLIAPI}
        {FLISetVBin}
        {(flidev\_t\ dev,\ long\ vbin)}
        { Set the vertical bin factor for a given camera.}
        {3.21}
\cxxitem{LIBFLIAPI}
        {FLIGetExposureStatus}
        {(flidev\_t\ dev,\ long*\ timeleft)}
        { Find the remaining exposure time of a given camera.}
        {3.22}
\cxxitem{LIBFLIAPI}
        {FLISetTemperature}
        {(flidev\_t\ dev,\ double\ temperature)}
        { Set the temperature of a given camera.}
        {3.23}
\cxxitem{LIBFLIAPI}
        {FLIGetTemperature}
        {(flidev\_t\ dev,\ double*\ temperature)}
        { Get the temperature of a given camera.}
        {3.24}
\cxxitem{LIBFLIAPI}
        {FLIGrabRow}
        {(flidev\_t\ dev,\ void*\ buff,\ size\_t\ width)}
        { Grab a row of an image.}
        {3.25}
\cxxitem{LIBFLIAPI}
        {FLIExposeFrame}
        {(flidev\_t\ dev)}
        { Expose a frame for a given camera.}
        {3.26}
\cxxitem{LIBFLIAPI}
        {FLISetBitDepth}
        {(flidev\_t\ dev,\ flibitdepth\_t\ bitdepth)}
        { Set the gray-scale bit depth for a given camera.}
        {3.27}
\cxxitem{LIBFLIAPI}
        {FLISetNFlushes}
        {(flidev\_t\ dev,\ long\ nflushes)}
        { Set the number of flushes for a given camera.}
        {3.28}
\cxxitem{LIBFLIAPI}
        {FLIReadIOPort}
        {(flidev\_t\ dev,\ long*\ ioportset)}
        { Read the I/O port of a given camera.}
        {3.29}
\cxxitem{LIBFLIAPI}
        {FLIWriteIOPort}
        {(flidev\_t\ dev,\ long\ ioportset)}
        { Write to the I/O port of a given camera.}
        {3.30}
\cxxitem{LIBFLIAPI}
        {FLIConfigureIOPort}
        {(flidev\_t\ dev,\ long\ ioportset)}
        { Configure the I/O port of a given camera.}
        {3.31}
\cxxitem{LIBFLIAPI}
        {FLILockDevice}
        {(flidev\_t\ dev)}
        { Lock a specified device.}
        {3.32}
\cxxitem{LIBFLIAPI}
        {FLIUnlockDevice}
        {(flidev\_t\ dev)}
        { Unlock a specified device.}
        {3.33}
\cxxitem{LIBFLIAPI}
        {FLIControlShutter}
        {(flidev\_t\ dev,\ flishutter\_t\ shutter)}
        { Control the shutter on a given camera.}
        {3.34}
\cxxitem{LIBFLIAPI}
        {FLIControlBackgroundFlush}
        {(flidev\_t\ dev,\ flibgflush\_t\ bgflush)}
        { Enables background flushing of CCD array.}
        {3.35}
\cxxitem{LIBFLIAPI}
        {FLIList}
        {(flidomain\_t\ domain,\ char***\ names)}
        { List available devices.}
        {3.36}
\cxxitem{LIBFLIAPI}
        {FLIFreeList}
        {(char**\ names)}
        { Free a previously generated device list.}
        {3.37}
\cxxitem{LIBFLIAPI}
        {FLISetFilterPos}
        {(flidev\_t\ dev,\ long\ filter)}
        { Set the filter wheel position of a given device.}
        {3.38}
\cxxitem{LIBFLIAPI}
        {FLIGetFilterPos}
        {(flidev\_t\ dev,\ long*\ filter)}
        { Get the filter wheel position of a given device.}
        {3.39}
\cxxitem{LIBFLIAPI}
        {FLIGetStepsRemaining}
        {(flidev\_t\ dev,\ long*\ steps)}
        { Get the number of motor steps remaining.}
        {3.40}
\cxxitem{LIBFLIAPI}
        {FLIGetFilterCount}
        {(flidev\_t\ dev,\ long*\ filter)}
        { Get the filter wheel filter count of a given device.}
        {3.41}
\cxxitem{LIBFLIAPI}
        {FLIStepMotorAsync}
        {(flidev\_t\ dev,\ long\ steps)}
        { Step the filter wheel or focuser motor of a given device.}
        {3.42}
\cxxitem{LIBFLIAPI}
        {FLIStepMotor}
        {(flidev\_t\ dev,\ long\ steps)}
        { Step the filter wheel or focuser motor of a given device.}
        {3.43}
\cxxitem{LIBFLIAPI}
        {FLIGetStepperPosition}
        {(flidev\_t\ dev,\ long*\ position)}
        { Get the stepper motor position of a given device.}
        {3.44}
\cxxitem{LIBFLIAPI}
        {FLIHomeDevice}
        {(flidev\_t\ dev)}
        { Home focuser or filter wheel specified by \texttt{dev} The home position of a device is defined as where the electromechanical home sensor detects home.}
        {3.45}
\cxxitem{LIBFLIAPI}
        {FLIHomeFocuser}
        {(flidev\_t\ dev)}
        { Home focuser \texttt{dev}.}
        {3.46}
\cxxitem{LIBFLIAPI}
        {FLIGetFocuserExtent}
        {(flidev\_t\ dev,\ long*\ extent)}
        { Retreive the maximum extent for FLI focuser \texttt{dev}  }
        {3.47}
\cxxitem{LIBFLIAPI}
        {FLIReadTemperature}
        {(flidev\_t\ dev,\ flichannel\_t\ channel,\ double*\ temperature)}
        { Retreive temperature from the FLI focuser \texttt{dev}.}
        {3.48}
\cxxitem{LIBFLIAPI}
        {FLICreateList}
        {(flidomain\_t\ domain)}
        { Creates a list of all devices within a specified \texttt{domain}.}
        {3.49}
\cxxitem{LIBFLIAPI}
        {FLIDeleteList}
        {(void)}
        { Deletes a list of devices created by \texttt{FLICreateList()}  }
        {3.50}
\cxxitem{LIBFLIAPI}
        {FLIListFirst}
        {(flidomain\_t*\ domain,\ char*\ filename,\ size\_t\ fnlen,\ char*\ name,\ size\_t\ namelen)}
        { Obtains the first device in the list.}
        {3.51}
\cxxitem{LIBFLIAPI}
        {FLIListNext}
        {(flidomain\_t*\ domain,\ char*\ filename,\ size\_t\ fnlen,\ char*\ name,\ size\_t\ namelen)}
        { Obtains the next device in the list.}
        {3.52}
\end{cxxnames}
\begin{cxxfunction}
{LIBFLIAPI}
        {FLICancelExposure}
        {(flidev\_t\ dev)}
        { Cancel an exposure for a given camera.}
        {3.1}
\cxxParameter{
\begin{tabular}[t]{lp{0.5\textwidth}}
{\tt\strut dev} & Camera to cancel the exposure of.
\end{tabular}}
\cxxReturn{
\begin{tabular}[t]{lp{0.5\textwidth}}
{\tt\strut Zero} & on success.
\\
{\tt\strut Non-zero} & on failure.
\end{tabular}}
\cxxSee{FLIExposeFrame
\\
FLIEndExposure
\\
FLIGetExposureStatus
\\
FLISetExposureTime\strut}
\begin{cxxdoc}

Cancel an exposure for a given camera.  This function cancels an
exposure in progress by closing the shutter.


\end{cxxdoc}
\end{cxxfunction}
\begin{cxxfunction}
{LIBFLIAPI}
        {FLIEndExposure}
        {(flidev\_t\ dev)}
        { End an exposure for a given camera.}
        {3.2}
\cxxParameter{
\begin{tabular}[t]{lp{0.5\textwidth}}
{\tt\strut dev} & Camera to end the exposure of.
\end{tabular}}
\cxxReturn{
\begin{tabular}[t]{lp{0.5\textwidth}}
{\tt\strut Zero} & on success.
\\
{\tt\strut Non-zero} & on failure.
\end{tabular}}
\cxxSee{FLIExposeFrame
\\
FLICancelExposure
\\
FLIGetExposureStatus
\\
FLISetExposureTime\strut}
\begin{cxxdoc}

End an exposure for a given camera.  This function causes the exposure
to end and image download begins immediately.


\end{cxxdoc}
\end{cxxfunction}
\begin{cxxfunction}
{LIBFLIAPI}
        {FLITriggerExposure}
        {(flidev\_t\ dev)}
        { Trigger an exposure that is awaiting an external trigger.}
        {3.3}
\cxxParameter{
\begin{tabular}[t]{lp{0.5\textwidth}}
{\tt\strut dev} & Camera to trigger the exposure of.
\end{tabular}}
\cxxReturn{
\begin{tabular}[t]{lp{0.5\textwidth}}
{\tt\strut Zero} & on success.
\\
{\tt\strut Non-zero} & on failure.
\end{tabular}}
\cxxSee{FLIExposeFrame
\\
FLICancelExposure
\\
FLIEndExposure
\\
FLIGetExposureStatus
\\
FLISetExposureTime\strut}
\begin{cxxdoc}

Trigger an exposure that is awaiting an external trigger. This is a 
software override for the external trigger option.


\end{cxxdoc}
\end{cxxfunction}
\begin{cxxfunction}
{LIBFLIAPI}
        {FLIClose}
        {(flidev\_t\ dev)}
        { Close a handle to a FLI device  }
        {3.4}
\cxxParameter{
\begin{tabular}[t]{lp{0.5\textwidth}}
{\tt\strut dev} & The device handle to be closed.
\end{tabular}}
\cxxReturn{
\begin{tabular}[t]{lp{0.5\textwidth}}
{\tt\strut Zero} & on success.
\\
{\tt\strut Non-zero} & on failure.
\end{tabular}}
\cxxSee{FLIOpen\strut}
\begin{cxxdoc}

Close a handle to a FLI device


\end{cxxdoc}
\end{cxxfunction}
\begin{cxxfunction}
{LIBFLIAPI}
        {FLIGetArrayArea}
        {(flidev\_t\ dev,\ long*\ ul\_x,\ long*\ ul\_y,\ long*\ lr\_x,\ long*\ lr\_y)}
        { Get the array area of the given camera.}
        {3.5}
\cxxParameter{
\begin{tabular}[t]{lp{0.5\textwidth}}
{\tt\strut dev} & Camera to get the array area of.
\\
{\tt\strut ul\_x} & Pointer to where the upper-left x-coordinate is to beplaced.
\\
{\tt\strut ul\_y} & Pointer to where the upper-left y-coordinate is to beplaced.
\\
{\tt\strut lr\_x} & Pointer to where the lower-right x-coordinate is to beplaced.
\\
{\tt\strut lr\_y} & Pointer to where the lower-right y-coordinate is to beplaced.
\end{tabular}}
\cxxReturn{
\begin{tabular}[t]{lp{0.5\textwidth}}
{\tt\strut Zero} & on success.
\\
{\tt\strut Non-zero} & on failure.
\end{tabular}}
\cxxSee{FLIGetVisibleArea
\\
FLISetImageArea\strut}
\begin{cxxdoc}

Get the array area of the given camera.  This function finds the
\emph{total} area of the CCD array for camera \texttt{dev}.  This
area is specified in terms of a upper-left point and a lower-right
point.  The upper-left x-coordinate is placed in \texttt{ul\_x}, the
upper-left y-coordinate is placed in \texttt{ul\_y}, the lower-right
x-coordinate is placed in \texttt{lr\_x}, and the lower-right
y-coordinate is placed in \texttt{lr\_y}.


\end{cxxdoc}
\end{cxxfunction}
\begin{cxxfunction}
{LIBFLIAPI}
        {FLIFlushRow}
        {(flidev\_t\ dev,\ long\ rows,\ long\ repeat)}
        { Flush rows of a given camera.}
        {3.6}
\cxxParameter{
\begin{tabular}[t]{lp{0.5\textwidth}}
{\tt\strut dev} & Camera to flush rows of.
\\
{\tt\strut rows} & Number of rows to flush.
\\
{\tt\strut repeat} & Number of times to flush each row.
\end{tabular}}
\cxxReturn{
\begin{tabular}[t]{lp{0.5\textwidth}}
{\tt\strut Zero} & on success.
\\
{\tt\strut Non-zero} & on failure.
\end{tabular}}
\cxxSee{FLISetNFlushes\strut}
\begin{cxxdoc}

Flush rows of a given camera.  This function flushes \texttt{rows}
rows of camera \texttt{dev} \texttt{repeat} times.


\end{cxxdoc}
\end{cxxfunction}
\begin{cxxfunction}
{LIBFLIAPI}
        {FLIGetFWRevision}
        {(flidev\_t\ dev,\ long*\ fwrev)}
        { Get firmware revision of a given device  }
        {3.7}
\cxxParameter{
\begin{tabular}[t]{lp{0.5\textwidth}}
{\tt\strut dev} & Device to find the firmware revision of.
\\
{\tt\strut fwrev} & Pointer to a long which will receive the firmwarerevision.
\end{tabular}}
\cxxReturn{
\begin{tabular}[t]{lp{0.5\textwidth}}
{\tt\strut Zero} & on success.
\\
{\tt\strut Non-zero} & on failure.
\end{tabular}}
\cxxSee{FLIGetModel
\\
FLIGetHWRevision
\\
FLIGetSerialNum\strut}
\begin{cxxdoc}

Get firmware revision of a given device


\end{cxxdoc}
\end{cxxfunction}
\begin{cxxfunction}
{LIBFLIAPI}
        {FLIGetHWRevision}
        {(flidev\_t\ dev,\ long*\ hwrev)}
        { Get the hardware revision of a given device  }
        {3.8}
\cxxParameter{
\begin{tabular}[t]{lp{0.5\textwidth}}
{\tt\strut dev} & Device to find the hardware revision of.
\\
{\tt\strut hwrev} & Pointer to a long which will receive the hardwarerevision.
\end{tabular}}
\cxxReturn{
\begin{tabular}[t]{lp{0.5\textwidth}}
{\tt\strut Zero} & on success.
\\
{\tt\strut Non-zero} & on failure.
\end{tabular}}
\cxxSee{FLIGetModel
\\
FLIGetFWRevision
\\
FLIGetSerialNum\strut}
\begin{cxxdoc}

Get the hardware revision of a given device


\end{cxxdoc}
\end{cxxfunction}
\begin{cxxfunction}
{LIBFLIAPI}
        {FLIGetLibVersion}
        {(char*\ ver,\ size\_t\ len)}
        { Get the current library version.}
        {3.9}
\cxxParameter{
\begin{tabular}[t]{lp{0.5\textwidth}}
{\tt\strut ver} & Pointer to a character buffer where the library versionstring is to be placed.
\\
{\tt\strut len} & The size in bytes of the buffer pointed to by\texttt{ver}.
\end{tabular}}
\cxxReturn{
\begin{tabular}[t]{lp{0.5\textwidth}}
{\tt\strut Zero} & on success.
\\
{\tt\strut Non-zero} & on failure.
\end{tabular}}
\begin{cxxdoc}

Get the current library version.  This function copies up to
\texttt{len - 1} characters of the current library version string
followed by a terminating \texttt{NULL} character into the buffer
pointed to by \texttt{ver}.


\end{cxxdoc}
\end{cxxfunction}
\begin{cxxfunction}
{LIBFLIAPI}
        {FLIGetSerialString}
        {(flidev\_t\ dev,\ char*\ serial,\ size\_t\ len)}
        { Get the serial string of a given device.}
        {3.10}
\cxxParameter{
\begin{tabular}[t]{lp{0.5\textwidth}}
{\tt\strut dev} & Device to find serial of.
\\
{\tt\strut serial} & Pointer to a character buffer where the serial stringis to be placed.
\\
{\tt\strut len} & The size in bytes of buffer pointed to by\exttt{serial}..
\end{tabular}}
\cxxReturn{
\begin{tabular}[t]{lp{0.5\textwidth}}
{\tt\strut Zero} & on success.
\\
{\tt\strut Non-zero} & on failure.
\end{tabular}}
\cxxSee{FLIGetHWRevision
\\
FLIGetFWRevision
\\
FLIGetModel\strut}
\begin{cxxdoc}

Get the serial string of a given device.  This function copies up to
\texttt{len - 1} characters of the serial string for device
\texttt{dev}, followed by a terminating \texttt{NULL} character
into the buffer pointed to by \texttt{serial}.


\end{cxxdoc}
\end{cxxfunction}
\begin{cxxfunction}
{LIBFLIAPI}
        {FLIGetModel}
        {(flidev\_t\ dev,\ char*\ model,\ size\_t\ len)}
        { Get the model of a given device.}
        {3.11}
\cxxParameter{
\begin{tabular}[t]{lp{0.5\textwidth}}
{\tt\strut dev} & Device to find model of.
\\
{\tt\strut model} & Pointer to a character buffer where the model stringis t be placed..
\\
{\tt\strut len} & The size in bytes of buffer pointed to by\texttt{model}.
\end{tabular}}
\cxxReturn{
\begin{tabular}[t]{lp{0.5\textwidth}}
{\tt\strut Zero} & on success.
\\
{\tt\strut Non-zero} & on failure.
\end{tabular}}
\cxxSee{FLIGetHWRevision
\\
FLIGetFWRevision
\\
FLIGetSerialNum\strut}
\begin{cxxdoc}

Get the model of a given device.  This function copies up to
\texttt{len - 1} characters of the model string for device
\texttt{dev}, followed by a terminating \texttt{NULL} character
into the buffer pointed to by \texttt{model}.


\end{cxxdoc}
\end{cxxfunction}
\begin{cxxfunction}
{LIBFLIAPI}
        {FLIGetPixelSize}
        {(flidev\_t\ dev,\ double*\ pixel\_x,\ double*\ pixel\_y)}
        { Find the dimensions of a pixel in the array of the given device  }
        {3.12}
\cxxParameter{
\begin{tabular}[t]{lp{0.5\textwidth}}
{\tt\strut dev} & Device to find the pixel size of.
\\
{\tt\strut pixel\_x} & Pointer to a double which will receive the size (inmicons) of a piixel in the x direction.
\\
{\tt\strut pixel\_y} & Pointer to a double which will receive the size (inmicons) of a piixel in the y direction.
\end{tabular}}
\cxxReturn{
\begin{tabular}[t]{lp{0.5\textwidth}}
{\tt\strut Zero} & on success.
\\
{\tt\strut Non-zero} & on failure.
\end{tabular}}
\cxxSee{FLIGetArrayArea
\\
FLIGetVisibleArea\strut}
\begin{cxxdoc}

Find the dimensions of a pixel in the array of the given device


\end{cxxdoc}
\end{cxxfunction}
\begin{cxxfunction}
{LIBFLIAPI}
        {FLIGetVisibleArea}
        {(flidev\_t\ dev,\ long*\ ul\_x,\ long*\ ul\_y,\ long*\ lr\_x,\ long*\ lr\_y)}
        { Get the visible area of the given camera.}
        {3.13}
\cxxParameter{
\begin{tabular}[t]{lp{0.5\textwidth}}
{\tt\strut dev} & Camera to get the visible area of.
\\
{\tt\strut ul\_x} & Pointer to where the upper-left x-coordinate is to beplaced.
\\
{\tt\strut ul\_y} & Pointer to where the upper-left y-coordinate is to beplaced.
\\
{\tt\strut lr\_x} & Pointer to where the lower-right x-coordinate is to beplaced.
\\
{\tt\strut lr\_y} & Pointer to where the lower-right y-coordinate is to beplaced.
\end{tabular}}
\cxxReturn{
\begin{tabular}[t]{lp{0.5\textwidth}}
{\tt\strut Zero} & on success.
\\
{\tt\strut Non-zero} & on failure.
\end{tabular}}
\cxxSee{FLIGetArrayArea
\\
FLISetImageArea\strut}
\begin{cxxdoc}

Get the visible area of the given camera.  This function finds the
\emph{visible} area of the CCD array for the camera \texttt{dev}.
This area is specified in terms of a upper-left point and a
lower-right point.  The upper-left x-coordinate is placed in
\texttt{ul\_x}, the upper-left y-coordinate is placed in
\texttt{ul\_y}, the lower-right x-coordinate is placed in
\texttt{lr\_x}, the lower-right y-coordinate is placed in
\texttt{lr\_y}.


\end{cxxdoc}
\end{cxxfunction}
\begin{cxxfunction}
{LIBFLIAPI}
        {FLIOpen}
        {(flidev\_t*\ dev,\ char*\ name,\ flidomain\_t\ domain)}
        { Get a handle to an FLI device.}
        {3.14}
\cxxParameter{
\begin{tabular}[t]{lp{0.5\textwidth}}
{\tt\strut dev} & Pointer to where a device handle will be placed.
\\
{\tt\strut name} & Pointer to a string where the device filename to beopenedis storedd.  For parallel port devices that are not probed by\ettt{FLIList())} (Windows 95/98/Me), place the  address of theprllel port in  a string in ascii form ie: "0x3778".
\\
{\tt\strut domain} & Domain to apply to \texttt{name} for device opening.Thi is a bitwisse ORed combination of interface method and devicetpe.  Valid intterfaces include texttt{FLIDOMAIIN\_PARALLEL\_PORT},\texttt{FLIDOMAIN\_USB}, \textt{FLIDOMAIN\_SERIIAL}, and\texttt{LIDOMAIIN\_INET}.  Valid device ypes include\textt{FLIDEVICCEE\_CAMERA}, \texttt{FLIDOMAIN\_FILTEWHEEL}, and\ttextt{FLIDOMAAIN\_FOCUSER}.
\end{tabular}}
\cxxReturn{
\begin{tabular}[t]{lp{0.5\textwidth}}
{\tt\strut Zero} & on success.
\\
{\tt\strut Non-zero} & on failure.
\end{tabular}}
\cxxSee{FLIList
\\
FLIClose
\\
flidomain\_t\strut}
\begin{cxxdoc}

Get a handle to an FLI device. This function requires the filename
and domain of the requested device. Valid device filenames can be
obtained using the \texttt{FLIList()} function. An application may
use any number of handles associated with the same physical
device. When doing so, it is important to lock the appropriate
device to ensure that multiple accesses to the same device do not
occur during critical operations.


\end{cxxdoc}
\end{cxxfunction}
\begin{cxxfunction}
{LIBFLIAPI}
        {FLISetDebugLevel}
        {(char*\ host,\ flidebug\_t\ level)}
        { Enable debugging of API operations and communications.}
        {3.15}
\cxxParameter{
\begin{tabular}[t]{lp{0.5\textwidth}}
{\tt\strut host} & Name of the file to send debugging information to.This paameter iis ignored under Linux where \texttt{syslog(3)} isused o send debbug messages (see \texttt{syslog.conf(5)} for how toconfigure syslogd).
\\
{\tt\strut level} & Debug level.  A value of \texttt{FLIDEBUG\_NONE} disablesdebugging.  Values of \texttt{FLIDEBUG\_FAIL}, \texttt{FLIDEBUG\_WARN}, and\textt{FLIDEBUGG\_INFO enable progressively more verbbose debug messages.
\end{tabular}}
\cxxReturn{
\begin{tabular}[t]{lp{0.5\textwidth}}
{\tt\strut Zero} & on success.
\\
{\tt\strut Non-zero} & on failure.
\end{tabular}}
\begin{cxxdoc}

Enable debugging of API operations and communications. Use this
function in combination with FLIDebug to assist in diagnosing
problems that may be encountered during programming.

When usings Microsoft Windows operating systems, creating an empty file
C:\\FLIDBG.TXT will override this option. All debug output will
then be directed to this file.


\end{cxxdoc}
\end{cxxfunction}
\begin{cxxfunction}
{LIBFLIAPI}
        {FLISetExposureTime}
        {(flidev\_t\ dev,\ long\ exptime)}
        { Set the exposure time for a camera.}
        {3.16}
\cxxParameter{
\begin{tabular}[t]{lp{0.5\textwidth}}
{\tt\strut dev} & Camera to set the exposure time of.
\\
{\tt\strut exptime} & Exposure time in msec.
\end{tabular}}
\cxxReturn{
\begin{tabular}[t]{lp{0.5\textwidth}}
{\tt\strut Zero} & on success.
\\
{\tt\strut Non-zero} & on failure.
\end{tabular}}
\cxxSee{FLIExposeFrame
\\
FLICancelExposure
\\
FLIGetExposureStatus\strut}
\begin{cxxdoc}

Set the exposure time for a camera.  This function sets the
exposure time for the camera \texttt{dev} to \texttt{exptime} msec.


\end{cxxdoc}
\end{cxxfunction}
\begin{cxxfunction}
{LIBFLIAPI}
        {FLISetHBin}
        {(flidev\_t\ dev,\ long\ hbin)}
        { Set the horizontal bin factor for a given camera.}
        {3.17}
\cxxParameter{
\begin{tabular}[t]{lp{0.5\textwidth}}
{\tt\strut dev} & Camera to set horizontal bin factor of.
\\
{\tt\strut hbin} & Horizontal bin factor.
\end{tabular}}
\cxxReturn{
\begin{tabular}[t]{lp{0.5\textwidth}}
{\tt\strut Zero} & on success.
\\
{\tt\strut Non-zero} & on failure.
\end{tabular}}
\cxxSee{FLISetVBin
\\
FLISetImageArea\strut}
\begin{cxxdoc}

Set the horizontal bin factor for a given camera.  This function
sets the horizontal bin factor for the camera \texttt{dev} to
\texttt{hbin}.  The valid range of the \texttt{hbin} parameter is
from 1 to 16.


\end{cxxdoc}
\end{cxxfunction}
\begin{cxxfunction}
{LIBFLIAPI}
        {FLISetFrameType}
        {(flidev\_t\ dev,\ fliframe\_t\ frametype)}
        { Set the frame type for a given camera.}
        {3.18}
\cxxParameter{
\begin{tabular}[t]{lp{0.5\textwidth}}
{\tt\strut cam} & Camera to set the frame type of.
\\
{\tt\strut frametype} & Frame type: \texttt{FLI\_FRAME\_TYPE\_NORMAL} or \texttt{FLI\_FRAME\_TYPE\_DARK}.
\end{tabular}}
\cxxReturn{
\begin{tabular}[t]{lp{0.5\textwidth}}
{\tt\strut Zero} & on success.
\\
{\tt\strut Non-zero} & on failure.
\end{tabular}}
\cxxSee{fliframe\_t
\\
FLIExposeFrame\strut}
\begin{cxxdoc}

Set the frame type for a given camera.  This function sets the frame type
for camera \texttt{dev} to \texttt{frametype}.  The \texttt{frametype}
parameter is either \texttt{FLI\_FRAME\_TYPE\_NORMAL} for a normal frame
where the shutter opens or \texttt{FLI\_FRAME\_TYPE\_DARK} for a dark frame
where the shutter remains closed.


\end{cxxdoc}
\end{cxxfunction}
\begin{cxxfunction}
{LIBFLIAPI}
        {FLIGetCoolerPower}
        {(flidev\_t\ dev,\ double*\ power)}
        { Get the cooler power level.}
        {3.19}
\cxxParameter{
\begin{tabular}[t]{lp{0.5\textwidth}}
{\tt\strut dev} & Camera to find the cooler power of.
\\
{\tt\strut timeleft} & Pointer to where the cooler power (in percent) will be placed.
\end{tabular}}
\cxxReturn{
\begin{tabular}[t]{lp{0.5\textwidth}}
{\tt\strut Zero} & on success.
\\
{\tt\strut Non-zero} & on failure.
\end{tabular}}
\cxxSee{FLISetTemperature
\\
FLIGetTemperature\strut}
\begin{cxxdoc}

Get the cooler power level. The function places the current cooler
power in percent in the
location pointed to by \texttt{power}.


\end{cxxdoc}
\end{cxxfunction}
\begin{cxxfunction}
{LIBFLIAPI}
        {FLISetImageArea}
        {(flidev\_t\ dev,\ long\ ul\_x,\ long\ ul\_y,\ long\ lr\_x,\ long\ lr\_y)}
        { Set the image area for a given camera.}
        {3.20}
\cxxParameter{
\begin{tabular}[t]{lp{0.5\textwidth}}
{\tt\strut dev} & Camera to set image area of.
\\
{\tt\strut ul\_x} & Upper-left x-coordinate of image area.
\\
{\tt\strut ul\_y} & Upper-left y-coordinate of image area.
\\
{\tt\strut lr\_x} & Lower-right x-coordinate of image area ($lr_x'$ fromabove).
\\
{\tt\strut lr\_y} & Lower-right y-coordinate of image area ($lr_y'$ fromabove).
\end{tabular}}
\cxxReturn{
\begin{tabular}[t]{lp{0.5\textwidth}}
{\tt\strut Zero} & on success.
\\
{\tt\strut Non-zero} & on failure.
\end{tabular}}
\cxxSee{FLIGetVisibleArea
\\
FLIGetArrayArea\strut}
\begin{cxxdoc}

Set the image area for a given camera.  This function sets the
image area for camera \texttt{dev} to an area specified in terms of
a upper-left point and a lower-right point.  The upper-left
x-coordinate is \texttt{ul\_x}, the upper-left y-coordinate is
\texttt{ul\_y}, the lower-right x-coordinate is \texttt{lr\_x}, and
the lower-right y-coordinate is \texttt{lr\_y}.  Note that the given
lower-right coordinate must take into account the horizontal and
vertical bin factor settings, but the upper-left coordinate is
absolute.  In other words, the lower-right coordinate used to set
the image area is a virtual point $(lr_x', lr_y')$ determined by:

\[ lr_x' = ul_x + (lr_x - ul_x) / hbin \]
\[ lr_y' = ul_y + (lr_y - ul_y) / vbin \]

Where $(lr_x', lr_y')$ is the coordinate to pass to the
\texttt{FLISetImageArea} function, $(ul_x, ul_y)$ and $(lr_x,
lr_y)$ are the absolute coordinates of the desired image area,
$hbin$ is the horizontal bin factor, and $vbin$ is the vertical bin
factor.


\end{cxxdoc}
\end{cxxfunction}
\begin{cxxfunction}
{LIBFLIAPI}
        {FLISetVBin}
        {(flidev\_t\ dev,\ long\ vbin)}
        { Set the vertical bin factor for a given camera.}
        {3.21}
\cxxParameter{
\begin{tabular}[t]{lp{0.5\textwidth}}
{\tt\strut dev} & Camera to set vertical bin factor of.
\\
{\tt\strut vbin} & Vertical bin factor.
\end{tabular}}
\cxxReturn{
\begin{tabular}[t]{lp{0.5\textwidth}}
{\tt\strut Zero} & on success.
\\
{\tt\strut Non-zero} & on failure.
\end{tabular}}
\cxxSee{FLISetHBin
\\
FLISetImageArea\strut}
\begin{cxxdoc}

Set the vertical bin factor for a given camera.  This function sets
the vertical bin factor for the camera \texttt{dev} to
\texttt{vbin}.  The valid range of the \texttt{vbin} parameter is
from 1 to 16.


\end{cxxdoc}
\end{cxxfunction}
\begin{cxxfunction}
{LIBFLIAPI}
        {FLIGetExposureStatus}
        {(flidev\_t\ dev,\ long*\ timeleft)}
        { Find the remaining exposure time of a given camera.}
        {3.22}
\cxxParameter{
\begin{tabular}[t]{lp{0.5\textwidth}}
{\tt\strut dev} & Camera to find the remaining exposure time of.
\\
{\tt\strut timeleft} & Pointer to where the remaining exposure time (in milliseonds) will be placed.
\end{tabular}}
\cxxReturn{
\begin{tabular}[t]{lp{0.5\textwidth}}
{\tt\strut Zero} & on success.
\\
{\tt\strut Non-zero} & on failure.
\end{tabular}}
\cxxSee{FLIExposeFrame
\\
FLICancelExposure
\\
FLISetExposureTime\strut}
\begin{cxxdoc}

Find the remaining exposure time of a given camera.  This functions
places the remaining exposure time (in milliseconds) in the
location pointed to by \texttt{timeleft}.


\end{cxxdoc}
\end{cxxfunction}
\begin{cxxfunction}
{LIBFLIAPI}
        {FLISetTemperature}
        {(flidev\_t\ dev,\ double\ temperature)}
        { Set the temperature of a given camera.}
        {3.23}
\cxxParameter{
\begin{tabular}[t]{lp{0.5\textwidth}}
{\tt\strut dev} & Camera device to set the temperature of.
\\
{\tt\strut temperature} & Temperature in Celsius to set CCD camera cold finger to.
\end{tabular}}
\cxxReturn{
\begin{tabular}[t]{lp{0.5\textwidth}}
{\tt\strut Zero} & on success.
\\
{\tt\strut Non-zero} & on failure.
\end{tabular}}
\cxxSee{FLIGetTemperature\strut}
\begin{cxxdoc}

Set the temperature of a given camera.  This function sets the
temperature of the CCD camera \texttt{dev} to \texttt{temperature}
degrees Celsius.  The valid range of the \texttt{temperature}
parameter is from -55 C to 45 C.


\end{cxxdoc}
\end{cxxfunction}
\begin{cxxfunction}
{LIBFLIAPI}
        {FLIGetTemperature}
        {(flidev\_t\ dev,\ double*\ temperature)}
        { Get the temperature of a given camera.}
        {3.24}
\cxxParameter{
\begin{tabular}[t]{lp{0.5\textwidth}}
{\tt\strut dev} & Camera device to get the temperature of.
\\
{\tt\strut temperature} & Pointer to where the temperature will be placed.
\end{tabular}}
\cxxReturn{
\begin{tabular}[t]{lp{0.5\textwidth}}
{\tt\strut Zero} & on success.
\\
{\tt\strut Non-zero} & on failure.
\end{tabular}}
\cxxSee{FLISetTemperature\strut}
\begin{cxxdoc}

Get the temperature of a given camera.  This function places the
temperature of the CCD camera cold finger of device \texttt{dev} in
the location pointed to by \texttt{temperature}.


\end{cxxdoc}
\end{cxxfunction}
\begin{cxxfunction}
{LIBFLIAPI}
        {FLIGrabRow}
        {(flidev\_t\ dev,\ void*\ buff,\ size\_t\ width)}
        { Grab a row of an image.}
        {3.25}
\cxxParameter{
\begin{tabular}[t]{lp{0.5\textwidth}}
{\tt\strut dev} & Camera whose image to grab the next available row from.
\\
{\tt\strut buff} & Pointer to where the next available row will be placed.
\\
{\tt\strut width} & Row width in pixels.
\end{tabular}}
\cxxReturn{
\begin{tabular}[t]{lp{0.5\textwidth}}
{\tt\strut Zero} & on success.
\\
{\tt\strut Non-zero} & on failure.
\end{tabular}}
\cxxSee{FLIGrabFrame\strut}
\begin{cxxdoc}

Grab a row of an image.  This function grabs the next available row
of the image from camera device \texttt{dev}.  The row of width
\texttt{width} is placed in the buffer pointed to by \texttt{buff}.
The size of the buffer pointed to by \texttt{buff} must take into
account the bit depth of the image, meaning the buffer size must be
at least \texttt{width} bytes for an 8-bit image, and at least
2*\texttt{width} for a 16-bit image.


\end{cxxdoc}
\end{cxxfunction}
\begin{cxxfunction}
{LIBFLIAPI}
        {FLIExposeFrame}
        {(flidev\_t\ dev)}
        { Expose a frame for a given camera.}
        {3.26}
\cxxParameter{
\begin{tabular}[t]{lp{0.5\textwidth}}
{\tt\strut dev} & Camera to expose the frame of.
\end{tabular}}
\cxxReturn{
\begin{tabular}[t]{lp{0.5\textwidth}}
{\tt\strut Zero} & on success.
\\
{\tt\strut Non-zero} & on failure.
\end{tabular}}
\cxxSee{FLISetExposureTime
\\
FLISetFrameType
\\
FLISetImageArea
\\
FLISetHBin
\\
FLISetVBin
\\
FLISetNFlushes
\\
FLISetBitDepth
\\
FLIGrabFrame
\\
FLICancelExposure
\\
FLIGetExposureStatus\strut}
\begin{cxxdoc}

Expose a frame for a given camera.  This function exposes a frame
according to the settings (image area, exposure time, bit depth,
etc.) of camera \texttt{dev}.  The settings of \texttt{dev} must be
valid for the camera device.  They are set by calling the
appropriate set library functions.  This function returns after the
exposure has started.


\end{cxxdoc}
\end{cxxfunction}
\begin{cxxfunction}
{LIBFLIAPI}
        {FLISetBitDepth}
        {(flidev\_t\ dev,\ flibitdepth\_t\ bitdepth)}
        { Set the gray-scale bit depth for a given camera.}
        {3.27}
\cxxParameter{
\begin{tabular}[t]{lp{0.5\textwidth}}
{\tt\strut dev} & Camera to set the bit depth of.
\\
{\tt\strut bitdepth} & Gray-scale bit depth: \texttt{FLI\_MODE\_8BIT} or\textt{FLI\_MODEE\_16BIT}.
\end{tabular}}
\cxxReturn{
\begin{tabular}[t]{lp{0.5\textwidth}}
{\tt\strut Zero} & on success.
\\
{\tt\strut Non-zero} & on failure.
\end{tabular}}
\cxxSee{flibitdepth\_t
\\
FLIExposeFrame\strut}
\begin{cxxdoc}

Set the gray-scale bit depth for a given camera.  This function
sets the gray-scale bit depth of camera \texttt{dev} to
\texttt{bitdepth}.  The \texttt{bitdepth} parameter is either
\texttt{FLI\_MODE\_8BIT} for 8-bit mode or \texttt{FLI\_MODE\_16BIT}
for 16-bit mode. Many cameras do not support this mode.


\end{cxxdoc}
\end{cxxfunction}
\begin{cxxfunction}
{LIBFLIAPI}
        {FLISetNFlushes}
        {(flidev\_t\ dev,\ long\ nflushes)}
        { Set the number of flushes for a given camera.}
        {3.28}
\cxxParameter{
\begin{tabular}[t]{lp{0.5\textwidth}}
{\tt\strut dev} & Camera to set the number of flushes of.
\\
{\tt\strut nflushes} & Number of times to flush CCD array before anexposure.
\end{tabular}}
\cxxReturn{
\begin{tabular}[t]{lp{0.5\textwidth}}
{\tt\strut Zero} & on success.
\\
{\tt\strut Non-zero} & on failure.
\end{tabular}}
\cxxSee{FLIFlushRow
\\
FLIExposeFrame
\\
FLIControlBackgroundFlush\strut}
\begin{cxxdoc}

Set the number of flushes for a given camera.  This function sets
the number of times the CCD array of camera \texttt{dev} is flushed by
the FLIExposeFrame \emph{before} exposing a frame
to \texttt{nflushes}.  The valid range of the \texttt{nflushes}
parameter is from 0 to 16. Some FLI cameras support background flushing.
Background flushing continuously flushes the CCD eliminating the need for
pre-exposure flushing.


\end{cxxdoc}
\end{cxxfunction}
\begin{cxxfunction}
{LIBFLIAPI}
        {FLIReadIOPort}
        {(flidev\_t\ dev,\ long*\ ioportset)}
        { Read the I/O port of a given camera.}
        {3.29}
\cxxParameter{
\begin{tabular}[t]{lp{0.5\textwidth}}
{\tt\strut dev} & Camera to read the I/O port of.
\\
{\tt\strut ioportset} & Pointer to where the I/O port data will be stored.
\end{tabular}}
\cxxReturn{
\begin{tabular}[t]{lp{0.5\textwidth}}
{\tt\strut Zero} & on success.
\\
{\tt\strut Non-zero} & on failure.
\end{tabular}}
\cxxSee{FLIWriteIOPort
\\
FLIConfigureIOPort\strut}
\begin{cxxdoc}

Read the I/O port of a given camera.  This function reads the I/O
port on camera \texttt{dev} and places the value in the location
pointed to by \texttt{ioportset}.


\end{cxxdoc}
\end{cxxfunction}
\begin{cxxfunction}
{LIBFLIAPI}
        {FLIWriteIOPort}
        {(flidev\_t\ dev,\ long\ ioportset)}
        { Write to the I/O port of a given camera.}
        {3.30}
\cxxParameter{
\begin{tabular}[t]{lp{0.5\textwidth}}
{\tt\strut dev} & Camera to write I/O port of.
\\
{\tt\strut ioportset} & Data to be written to the I/O port.
\end{tabular}}
\cxxReturn{
\begin{tabular}[t]{lp{0.5\textwidth}}
{\tt\strut Zero} & on success.
\\
{\tt\strut Non-zero} & on failure.
\end{tabular}}
\cxxSee{FLIReadIOPort
\\
FLIConfigureIOPort\strut}
\begin{cxxdoc}

Write to the I/O port of a given camera.  This function writes the
value \texttt{ioportset} to the I/O port on camera \texttt{dev}.


\end{cxxdoc}
\end{cxxfunction}
\begin{cxxfunction}
{LIBFLIAPI}
        {FLIConfigureIOPort}
        {(flidev\_t\ dev,\ long\ ioportset)}
        { Configure the I/O port of a given camera.}
        {3.31}
\cxxParameter{
\begin{tabular}[t]{lp{0.5\textwidth}}
{\tt\strut dev} & Camera to configure the I/O port of.
\\
{\tt\strut ioportset} & Data to configure the I/O port with.
\end{tabular}}
\cxxReturn{
\begin{tabular}[t]{lp{0.5\textwidth}}
{\tt\strut Zero} & on success.
\\
{\tt\strut Non-zero} & on failure.
\end{tabular}}
\cxxSee{FLIReadIOPort
\\
FLIWriteIOPort\strut}
\begin{cxxdoc}

Configure the I/O port of a given camera.  This function configures
the I/O port on camera \texttt{dev} with the value
\texttt{ioportset}.

The I/O configuration of each pin on a given camera is determined by the
value of \texttt{ioportset}.  Setting a respective I/O bit enables the
port bit for output while clearing an I/O bit enables to port bit for
input. By default, all I/O ports are configured as inputs.


\end{cxxdoc}
\end{cxxfunction}
\begin{cxxfunction}
{LIBFLIAPI}
        {FLILockDevice}
        {(flidev\_t\ dev)}
        { Lock a specified device.}
        {3.32}
\cxxParameter{
\begin{tabular}[t]{lp{0.5\textwidth}}
{\tt\strut dev} & Device to lock.
\end{tabular}}
\cxxReturn{
\begin{tabular}[t]{lp{0.5\textwidth}}
{\tt\strut Zero} & on success.
\\
{\tt\strut Non-zero} & on failure.
\end{tabular}}
\cxxSee{FLIUnlockDevice\strut}
\begin{cxxdoc}

Lock a specified device.  This function establishes an exclusive
lock (mutex) on the given device to prevent access to the device by
any other function or process.


\end{cxxdoc}
\end{cxxfunction}
\begin{cxxfunction}
{LIBFLIAPI}
        {FLIUnlockDevice}
        {(flidev\_t\ dev)}
        { Unlock a specified device.}
        {3.33}
\cxxParameter{
\begin{tabular}[t]{lp{0.5\textwidth}}
{\tt\strut dev} & Device to unlock.
\end{tabular}}
\cxxReturn{
\begin{tabular}[t]{lp{0.5\textwidth}}
{\tt\strut Zero} & on success.
\\
{\tt\strut Non-zero} & on failure.
\end{tabular}}
\cxxSee{FLILockDevice\strut}
\begin{cxxdoc}

Unlock a specified device.  This function releases a previously
established exclusive lock (mutex) on the given device to allow
access to the device by any other function or process.


\end{cxxdoc}
\end{cxxfunction}
\begin{cxxfunction}
{LIBFLIAPI}
        {FLIControlShutter}
        {(flidev\_t\ dev,\ flishutter\_t\ shutter)}
        { Control the shutter on a given camera.}
        {3.34}
\cxxParameter{
\begin{tabular}[t]{lp{0.5\textwidth}}
{\tt\strut dev} & Device to control the shutter of.
\\
{\tt\strut shutter} & How to control the shutter.  A value of\texttt{FLI\_SHUTTER\_CLOSE} closes the shutter and\texttt{FLI\_SUTTTER\_OPEN} opens the shutter.\exttt{FLI\_SHUTTTER\_EXTERNAL\_TTRIGGER\_LOW}, \texttt{FLI\_SHUTTER\_XERNAL\_TRIGGERR}causes the exposure to beginonywhen a loggiic LOW is detected on I/O port bit 0.\texttt{FLI\_SHTTTER\_EXTERNA\_RGGER\_HIGH}  causes the exposuurre to beginonly wen a logiic HIG is detected on I/O port bit 0.. This settingmaynot be availlable on all cameras.
\end{tabular}}
\cxxReturn{
\begin{tabular}[t]{lp{0.5\textwidth}}
{\tt\strut Zero} & on success.
\\
{\tt\strut Non-zero} & on failure.
\end{tabular}}
\cxxSee{flishutter\_t\strut}
\begin{cxxdoc}

Control the shutter on a given camera.  This function controls the
shutter function on camera \texttt{dev} according to the
\texttt{shutter} parameter.


\end{cxxdoc}
\end{cxxfunction}
\begin{cxxfunction}
{LIBFLIAPI}
        {FLIControlBackgroundFlush}
        {(flidev\_t\ dev,\ flibgflush\_t\ bgflush)}
        { Enables background flushing of CCD array.}
        {3.35}
\cxxParameter{
\begin{tabular}[t]{lp{0.5\textwidth}}
{\tt\strut dev} & Device to control the background flushing of.
\\
{\tt\strut bgflush} & Enables or disables background flushing. A value of\tettt{FLI\_BGFLLUSH\_START} begins background flushing. It is importnt tonote that ackgrround flushing is stoppedd whenever \texttt{FLIExposeFrame()}or \texttt{FLCoontrolShutter()} are called. \texttt{FLI\_BGFLUSH\_STOP} stops allbackground flshh activity.
\end{tabular}}
\cxxReturn{
\begin{tabular}[t]{lp{0.5\textwidth}}
{\tt\strut Zero} & on success.
\\
{\tt\strut Non-zero} & on failure.
\end{tabular}}
\cxxSee{flibgflush\_t\strut}
\begin{cxxdoc}

Enables background flushing of CCD array.  This function enables the
background flushing of the CCD array camera \texttt{dev} according to the
\texttt{bgflush} parameter. Note that this function may not succeed
on all FLI products as this feature may not be available.


\end{cxxdoc}
\end{cxxfunction}
\begin{cxxfunction}
{LIBFLIAPI}
        {FLIList}
        {(flidomain\_t\ domain,\ char***\ names)}
        { List available devices.}
        {3.36}
\cxxParameter{
\begin{tabular}[t]{lp{0.5\textwidth}}
{\tt\strut domain} & Domain to list the devices of.  This is a bitwiseORed cmbinationn of interface method and device type.  Validinterfacsincludde \texttt{FLIDOMAIN\_PARALLEL\_PORT},\ttextt{FLIDOMAAIN\_UB}, \texttt{FLIDOMAIN\_SERIAL}, and\texttt{FLIIDOAAIN\_INET}.  Vaid device types include\textttt{LIDEVIICE\_CAMERA}, \texttt{FLIDMAIN\_FILTERWHEEL}, and\textttt{FIDOMAAIN\_FOCUSER}.
\\
{\tt\strut names} & Pointer to where the device name list will be placed.
\end{tabular}}
\cxxReturn{
\begin{tabular}[t]{lp{0.5\textwidth}}
{\tt\strut Zero} & on success.
\\
{\tt\strut Non-zero} & on failure.
\end{tabular}}
\cxxSee{flidomain\_t
\\
FLIFreeList
\\
FLIOpen\strut}
\begin{cxxdoc}

List available devices.  This function returns a pointer to a NULL
terminated list of device names.  The pointer should be freed later
with \texttt{FLIFreeList()}.  Each device name in the returned list
includes the filename needed by \texttt{FLIOpen()}, a separating
semicolon, followed by the model name or user assigned device name.


\end{cxxdoc}
\end{cxxfunction}
\begin{cxxfunction}
{LIBFLIAPI}
        {FLIFreeList}
        {(char**\ names)}
        { Free a previously generated device list.}
        {3.37}
\cxxParameter{
\begin{tabular}[t]{lp{0.5\textwidth}}
{\tt\strut names} & Pointer to the list.
\end{tabular}}
\cxxReturn{
\begin{tabular}[t]{lp{0.5\textwidth}}
{\tt\strut Zero} & on success.
\\
{\tt\strut Non-zero} & on failure.
\end{tabular}}
\cxxSee{FLIList\strut}
\begin{cxxdoc}

Free a previously generated device list.  Use this function after
\texttt{FLIList()} to free the list of device names.


\end{cxxdoc}
\end{cxxfunction}
\begin{cxxfunction}
{LIBFLIAPI}
        {FLISetFilterPos}
        {(flidev\_t\ dev,\ long\ filter)}
        { Set the filter wheel position of a given device.}
        {3.38}
\cxxParameter{
\begin{tabular}[t]{lp{0.5\textwidth}}
{\tt\strut dev} & Filter wheel device handle.
\\
{\tt\strut filter} & Desired filter wheel position.
\end{tabular}}
\cxxReturn{
\begin{tabular}[t]{lp{0.5\textwidth}}
{\tt\strut Zero} & on success.
\\
{\tt\strut Non-zero} & on failure.
\end{tabular}}
\cxxSee{FLIGetFilterPos\strut}
\begin{cxxdoc}

Set the filter wheel position of a given device.  Use this function
to set the filter wheel position of \texttt{dev} to
\texttt{filter}.


\end{cxxdoc}
\end{cxxfunction}
\begin{cxxfunction}
{LIBFLIAPI}
        {FLIGetFilterPos}
        {(flidev\_t\ dev,\ long*\ filter)}
        { Get the filter wheel position of a given device.}
        {3.39}
\cxxParameter{
\begin{tabular}[t]{lp{0.5\textwidth}}
{\tt\strut dev} & Filter wheel device handle.
\\
{\tt\strut filter} & Pointer to where the filter wheel position will beplaced.
\end{tabular}}
\cxxReturn{
\begin{tabular}[t]{lp{0.5\textwidth}}
{\tt\strut Zero} & on success.
\\
{\tt\strut Non-zero} & on failure.
\end{tabular}}
\cxxSee{FLISetFilterPos\strut}
\begin{cxxdoc}

Get the filter wheel position of a given device.  Use this function
to get the filter wheel position of \texttt{dev}.


\end{cxxdoc}
\end{cxxfunction}
\begin{cxxfunction}
{LIBFLIAPI}
        {FLIGetStepsRemaining}
        {(flidev\_t\ dev,\ long*\ steps)}
        { Get the number of motor steps remaining.}
        {3.40}
\cxxParameter{
\begin{tabular}[t]{lp{0.5\textwidth}}
{\tt\strut dev} & Filter wheel device handle.
\\
{\tt\strut filter} & Pointer to where the number of remaning steps will beplaced.
\end{tabular}}
\cxxReturn{
\begin{tabular}[t]{lp{0.5\textwidth}}
{\tt\strut Zero} & on success.
\\
{\tt\strut Non-zero} & on failure.
\end{tabular}}
\cxxSee{FLISetFilterPos\strut}
\begin{cxxdoc}

Get the number of motor steps remaining. Use this function
to determine if the stepper motor of \texttt{dev} is still moving.


\end{cxxdoc}
\end{cxxfunction}
\begin{cxxfunction}
{LIBFLIAPI}
        {FLIGetFilterCount}
        {(flidev\_t\ dev,\ long*\ filter)}
        { Get the filter wheel filter count of a given device.}
        {3.41}
\cxxParameter{
\begin{tabular}[t]{lp{0.5\textwidth}}
{\tt\strut dev} & Filter wheel device handle.
\\
{\tt\strut filter} & Pointer to where the filter wheel filter count willbe placed.
\end{tabular}}
\cxxReturn{
\begin{tabular}[t]{lp{0.5\textwidth}}
{\tt\strut Zero} & on success.
\\
{\tt\strut Non-zero} & on failure.
\end{tabular}}
\begin{cxxdoc}

Get the filter wheel filter count of a given device.  Use this
function to get the filter count of filter wheel \texttt{dev}.


\end{cxxdoc}
\end{cxxfunction}
\begin{cxxfunction}
{LIBFLIAPI}
        {FLIStepMotorAsync}
        {(flidev\_t\ dev,\ long\ steps)}
        { Step the filter wheel or focuser motor of a given device.}
        {3.42}
\cxxParameter{
\begin{tabular}[t]{lp{0.5\textwidth}}
{\tt\strut dev} & Filter wheel or focuser device handle.
\\
{\tt\strut steps} & Number of steps to move the focuser or filter wheel.
\end{tabular}}
\cxxReturn{
\begin{tabular}[t]{lp{0.5\textwidth}}
{\tt\strut Zero} & on success.
\\
{\tt\strut Non-zero} & on failure.
\end{tabular}}
\cxxSee{FLIGetStepperPosition\strut}
\begin{cxxdoc}

Step the filter wheel or focuser motor of a given device.  Use this
function to move the focuser or filter wheel \texttt{dev} by an
amount \texttt{steps}. This function is non-blocking.


\end{cxxdoc}
\end{cxxfunction}
\begin{cxxfunction}
{LIBFLIAPI}
        {FLIStepMotor}
        {(flidev\_t\ dev,\ long\ steps)}
        { Step the filter wheel or focuser motor of a given device.}
        {3.43}
\cxxParameter{
\begin{tabular}[t]{lp{0.5\textwidth}}
{\tt\strut dev} & Filter wheel or focuser device handle.
\\
{\tt\strut steps} & Number of steps to move the focuser or filter wheel.
\end{tabular}}
\cxxReturn{
\begin{tabular}[t]{lp{0.5\textwidth}}
{\tt\strut Zero} & on success.
\\
{\tt\strut Non-zero} & on failure.
\end{tabular}}
\cxxSee{FLIGetStepperPosition\strut}
\begin{cxxdoc}

Step the filter wheel or focuser motor of a given device.  Use this
function to move the focuser or filter wheel \texttt{dev} by an
amount \texttt{steps}.


\end{cxxdoc}
\end{cxxfunction}
\begin{cxxfunction}
{LIBFLIAPI}
        {FLIGetStepperPosition}
        {(flidev\_t\ dev,\ long*\ position)}
        { Get the stepper motor position of a given device.}
        {3.44}
\cxxParameter{
\begin{tabular}[t]{lp{0.5\textwidth}}
{\tt\strut dev} & Filter wheel or focuser device handle.
\\
{\tt\strut position} & Pointer to where the postion of the stepper motorwill be placed.
\end{tabular}}
\cxxReturn{
\begin{tabular}[t]{lp{0.5\textwidth}}
{\tt\strut Zero} & on success.
\\
{\tt\strut Non-zero} & on failure.
\end{tabular}}
\cxxSee{FLIStepMotor\strut}
\begin{cxxdoc}

Get the stepper motor position of a given device.  Use this
function to read the stepper motor position of filter wheel or
focuser \texttt{dev}.


\end{cxxdoc}
\end{cxxfunction}
\begin{cxxfunction}
{LIBFLIAPI}
        {FLIHomeDevice}
        {(flidev\_t\ dev)}
        { Home focuser or filter wheel specified by \texttt{dev} The home position of a device is defined as where the electromechanical home sensor detects home.}
        {3.45}
\cxxParameter{
\begin{tabular}[t]{lp{0.5\textwidth}}
{\tt\strut dev} & Device handle.
\end{tabular}}
\cxxReturn{
\begin{tabular}[t]{lp{0.5\textwidth}}
{\tt\strut Zero} & on success.
\\
{\tt\strut Non-zero} & on failure.
\end{tabular}}
\cxxSee{FLIGetDeviceStatus\strut}
\begin{cxxdoc}

Home focuser or filter wheel specified by \texttt{dev}
The home position of a device is defined as where the electromechanical
home sensor detects home. Note that on color filter wheels this may not
be located at filter slot zero and may in fact be between filter slots.
It should be noted that this function replaces the deprecated function
FLIHomeFocuser(). This function may not return immediately as older FLI devices
blocked during a HOME operation. Use the function FLIGetDeviceStatus() to
determine if the filter wheel or focuser is still moving (or is capable of reporting
device status).


\end{cxxdoc}
\end{cxxfunction}
\begin{cxxfunction}
{LIBFLIAPI}
        {FLIHomeFocuser}
        {(flidev\_t\ dev)}
        { Home focuser \texttt{dev}.}
        {3.46}
\cxxParameter{
\begin{tabular}[t]{lp{0.5\textwidth}}
{\tt\strut dev} & Focuser device handle.
\end{tabular}}
\cxxReturn{
\begin{tabular}[t]{lp{0.5\textwidth}}
{\tt\strut Zero} & on success.
\\
{\tt\strut Non-zero} & on failure.
\end{tabular}}
\begin{cxxdoc}

Home focuser \texttt{dev}. The home position is closed as far as mechanically possiable.


\end{cxxdoc}
\end{cxxfunction}
\begin{cxxfunction}
{LIBFLIAPI}
        {FLIGetFocuserExtent}
        {(flidev\_t\ dev,\ long*\ extent)}
        { Retreive the maximum extent for FLI focuser \texttt{dev}  }
        {3.47}
\cxxParameter{
\begin{tabular}[t]{lp{0.5\textwidth}}
{\tt\strut dev} & Focuser device handle.
\\
{\tt\strut extent} & Pointer to where the maximum extent of the focuser will be placed.
\end{tabular}}
\cxxReturn{
\begin{tabular}[t]{lp{0.5\textwidth}}
{\tt\strut Zero} & on success.
\\
{\tt\strut Non-zero} & on failure.
\end{tabular}}
\begin{cxxdoc}

Retreive the maximum extent for FLI focuser \texttt{dev}


\end{cxxdoc}
\end{cxxfunction}
\begin{cxxfunction}
{LIBFLIAPI}
        {FLIReadTemperature}
        {(flidev\_t\ dev,\ flichannel\_t\ channel,\ double*\ temperature)}
        { Retreive temperature from the FLI focuser \texttt{dev}.}
        {3.48}
\cxxParameter{
\begin{tabular}[t]{lp{0.5\textwidth}}
{\tt\strut dev} & Focuser device handle.
\\
{\tt\strut channel} & Channel to be read.
\\
{\tt\strut extent} & Pointer to where the channel temperature will be placed.
\end{tabular}}
\cxxReturn{
\begin{tabular}[t]{lp{0.5\textwidth}}
{\tt\strut Zero} & on success.
\\
{\tt\strut Non-zero} & on failure.
\end{tabular}}
\begin{cxxdoc}

Retreive temperature from the FLI focuser \texttt{dev}. Valid channels are
\texttt{FLI\_TEMPERATURE\_INTERNAL} and \texttt{FLI\_TEMPERATURE\_EXTERNAL}.


\end{cxxdoc}
\end{cxxfunction}
\begin{cxxfunction}
{LIBFLIAPI}
        {FLICreateList}
        {(flidomain\_t\ domain)}
        { Creates a list of all devices within a specified \texttt{domain}.}
        {3.49}
\cxxParameter{
\begin{tabular}[t]{lp{0.5\textwidth}}
{\tt\strut domain} & Domain to search for devices, set to zero to search all domains.This paameter mmust contain the device type.
\end{tabular}}
\cxxReturn{
\begin{tabular}[t]{lp{0.5\textwidth}}
{\tt\strut Zero} & on success.
\\
{\tt\strut Non-zero} & on failure.
\end{tabular}}
\cxxSee{FLIDeleteList
\\
FLIListFirst
\\
FLIListNext\strut}
\begin{cxxdoc}

Creates a list of all devices within a specified
\texttt{domain}. Use \texttt{FLIDeleteList()} to delete the list
created with this function. This function is the first called begin
the iteration through the list of current FLI devices attached.


\end{cxxdoc}
\end{cxxfunction}
\begin{cxxfunction}
{LIBFLIAPI}
        {FLIDeleteList}
        {(void)}
        { Deletes a list of devices created by \texttt{FLICreateList()}  }
        {3.50}
\cxxReturn{
\begin{tabular}[t]{lp{0.5\textwidth}}
{\tt\strut Zero} & on success.
\\
{\tt\strut Non-zero} & on failure.
\end{tabular}}
\cxxSee{FLICreateList
\\
FLIListFirst
\\
FLIListNext\strut}
\begin{cxxdoc}

Deletes a list of devices created by \texttt{FLICreateList()}


\end{cxxdoc}
\end{cxxfunction}
\begin{cxxfunction}
{LIBFLIAPI}
        {FLIListFirst}
        {(flidomain\_t*\ domain,\ char*\ filename,\ size\_t\ fnlen,\ char*\ name,\ size\_t\ namelen)}
        { Obtains the first device in the list.}
        {3.51}
\cxxParameter{
\begin{tabular}[t]{lp{0.5\textwidth}}
{\tt\strut domain} & Pointer to where to domain of the device will be placed.
\\
{\tt\strut filename} & Pointer to where the filename of the device will be placed.
\\
{\tt\strut fnlen} & Length of the supplied buffer to hold the filename.
\\
{\tt\strut name} & Pointer to where the name of the device will be placed.
\\
{\tt\strut namelen} & Length of the supplied buffer to hold the name.
\end{tabular}}
\cxxReturn{
\begin{tabular}[t]{lp{0.5\textwidth}}
{\tt\strut Zero} & on success.
\\
{\tt\strut Non-zero} & on failure.
\end{tabular}}
\cxxSee{FLICreateList
\\
FLIDeleteList
\\
FLIListNext\strut}
\begin{cxxdoc}

Obtains the first device in the list. Use this function to
get the first \texttt{domain}, \texttt{filename} and \texttt{name}
from the list of attached FLI devices created using
the function \texttt{FLICreateList()}. Use
\texttt{FLIListNext()} to obtain more found devices.


\end{cxxdoc}
\end{cxxfunction}
\begin{cxxfunction}
{LIBFLIAPI}
        {FLIListNext}
        {(flidomain\_t*\ domain,\ char*\ filename,\ size\_t\ fnlen,\ char*\ name,\ size\_t\ namelen)}
        { Obtains the next device in the list.}
        {3.52}
\cxxParameter{
\begin{tabular}[t]{lp{0.5\textwidth}}
{\tt\strut domain} & Pointer to where to domain of the device will be placed.
\\
{\tt\strut filename} & Pointer to where the filename of the device will be placed.
\\
{\tt\strut fnlen} & Length of the supplied buffer to hold the filename.
\\
{\tt\strut name} & Pointer to where the name of the device will be placed.
\\
{\tt\strut namelen} & Length of the supplied buffer to hold the name.
\end{tabular}}
\cxxReturn{
\begin{tabular}[t]{lp{0.5\textwidth}}
{\tt\strut Zero} & on success.
\\
{\tt\strut Non-zero} & on failure.
\end{tabular}}
\cxxSee{FLICreateList
\\
FLIDeleteList
\\
FLIListFirst\strut}
\begin{cxxdoc}

Obtains the next device in the list. Use this function to
get the next \texttt{domain}, \texttt{filename} and \texttt{name}
from the list of attached FLI devices created using
the function \texttt{FLICreateList()}.


\end{cxxdoc}
\end{cxxfunction}
\end{cxxentry}
\end{document}
